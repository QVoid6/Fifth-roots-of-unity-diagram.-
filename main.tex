\documentclass[border=6pt, tikz]{standalone}

\usepackage{tikz}
  \usetikzlibrary{arrows.meta}
  \usetikzlibrary{shapes.geometric}
  \usetikzlibrary{calc}
\usepackage[margin=1in]{geometry}
\usepackage{xcolor}
  \pagecolor{black}
  \color{white}
\usepackage{contour}
  \contourlength{2pt}
\usepackage{pdfrender}
\usepackage{amsmath}
\usepackage{amssymb}
\usepackage{fontspec}
\usepackage{unicode-math}

\setmainfont{XITS}
\setmathfont{XITS Math}

\tikzset{axis/.style={-{Stealth}, thick}}
\tikzset{rootvec/.style={-{Latex}, ultra thick}}
\tikzset{sumvec1/.style={-{Latex}, thick}}
\tikzset{sumeqvec1/.style={dash pattern=on 1em off 0.5em, -{Latex}}}
\tikzset{sumvec2/.style={-{Latex}, gray}}
\tikzset{sumeqvec2/.style={ dash pattern=on 1em off 0.5em, -{Latex}, thin, gray}}
\tikzset{bi/.style={{Latex}-{Latex}, thick}}
\tikzset{cota/.style={midway, sloped, fill=black}}
\tikzset{coordmark/.style={dash pattern=on 0.5em off 0.5em, darkgray}}

\newcommand{\cota}[3]{\draw[bi] (#1) -- (#2)  node[cota, inner sep=2pt] {\small#3};}
\newcommand{\cotaround}[3]{\draw[bi] (#1) -- (#2)  node[cota, inner sep=1pt, ellipse] {\small#3};}

\newcommand{\xaxismark}[3]{\draw[thick] ({#1},-1pt) -- ({#1},1pt)  node[shift={#2}] {\small\textcolor{gray}{#3}};}
\newcommand{\yaxismark}[3]{\draw[thick] (-1pt,{#1}) -- (1pt,{#1})  node[shift={#2}] {\small\textcolor{gray}{#3}};}

\newcommand{\etiq}[2]{
    \node at (#1) {%
      \textpdfrender{
        TextRenderingMode=FillStrokeClip,
        LineWidth=6pt,
        FillColor=black,
        StrokeColor=black,
        MiterLimit=1
      }{#2}%
    };
    \node at (#1) {\textcolor{gray}{#2}};
}

\pgfmathsetmacro{\PHI}{(1+sqrt(5))/2}

\begin{document}

\begin{tikzpicture}[scale=5.04]
    % Coordenadas de las raíces
    \foreach \i in {0,...,4}{
      \coordinate (P\i) at (72*\i:1); 
    }

    % Coordenada del origen
    \coordinate (O) at (0,0);

    
    % Distancia entre raíces contiguas: |z^i - z^(i + 1)|
    \draw[very thin] (P0) \foreach \i in {1,...,4}{ -- (P\i)} -- cycle;

    % Distancia entre raíces diagonales: |z^i - z^(i + 2)|
    \foreach \i in {0,...,4}{
      \pgfmathtruncatemacro{\isuccsucc}{mod(\i+2,5)}
      \draw[bi, gray] (P\i) -- (P\isuccsucc);
    }

    % Distancia entre sumas de raíces contiguas: |(z^i + z^(i + 1)) - (z^(i + 1) + z^(i + 2))|
    \foreach \i in {0,...,4}{
      \pgfmathtruncatemacro{\isucc}{mod(\i+1,5)}
      \pgfmathtruncatemacro{\isuccsucc}{mod(\i+2,5)}
      \draw[bi, thin, gray] ($(P\i) + (P\isucc)$) -- ($(P\isucc) + (P\isuccsucc)$);
    }


    % Marcas de las coordenadas
    \foreach \i/\label in {0/a,1/b,2/c,3/d,4/e}{
    \path (P\i);
    \pgfgetlastxy{\Px\label}{\Py\label}
    
    \draw[coordmark] (P\i) -- (\Px\label/5.04,0);
    \draw[coordmark] (P\i) -- (0,\Py\label/5.04);
    }

    \foreach \i/\labelone/\labeltwo in {0/a/b,1/b/c,2/c/d,3/d/e,4/e/a}{
    \pgfmathtruncatemacro{\isucc}{mod(\i+1,5)}

    \path ($(P\i) + (P\isucc)$);
    \pgfgetlastxy{\Px\labelone\labeltwo}{\Py\labelone\labeltwo}
    
    \draw[coordmark] ($(P\i) + (P\isucc)$) -- (\Px\labelone\labeltwo/5.04,0);
    \draw[coordmark] ($(P\i) + (P\isucc)$) -- (0,\Py\labelone\labeltwo/5.04);
    }

    \foreach \i/\labelone/\labeltwo in {0/a/c,1/b/d,2/c/e,3/d/a,4/e/b}{
    \pgfmathtruncatemacro{\isuccsucc}{mod(\i+2,5)}

    \path ($(P\i) + (P\isuccsucc)$);
    \pgfgetlastxy{\Px\labelone\labeltwo}{\Py\labelone\labeltwo}
    
    \draw[coordmark] ($(P\i) + (P\isuccsucc)$) -- (\Px\labelone\labeltwo/5.04,0);
    \draw[coordmark] ($(P\i) + (P\isuccsucc)$) -- (0,\Py\labelone\labeltwo/5.04);
    }


    % Circunferencia unitaria
    \draw[thin] (O) circle (1);
    
    % Suma de raíces contiguas: z^i + z^(i + 1)
    \foreach \i in {0,...,4}{
      \pgfmathtruncatemacro{\isucc}{mod(\i+1,5)}
      \draw[sumeqvec1] (P\i) -- +(P\isucc);
      \draw[sumeqvec1] (P\isucc) -- +(P\i);
      
      \draw[sumvec1] (O) -- ($(P\i) + (P\isucc)$);
    }
    
    % Suma de raíces diagonales: z^i + z^(i + 2)
    \foreach \i in {0,...,4}{
      \pgfmathtruncatemacro{\isuccsucc}{mod(\i+2,5)}
      \draw[sumeqvec2] (P\i) -- +(P\isuccsucc);
      \draw[sumeqvec2] (P\isuccsucc) -- +(P\i);
      
      \draw[sumvec2] (O) -- ($(P\i) + (P\isuccsucc)$);
    }

    % Raíces: z^i
    \foreach \i in {0,...,4}{
      \draw[rootvec] (O) -- (72*\i:1);
    }


    % Ejes
    \draw[axis] (-2,0) -- (2,0)
      node[shift={(-0.2,0.3)}] {Re};
    \draw[axis] (0,-2) -- (0,2)
      node[shift={(0.35,-0.15)}] {Im};

    % Marcas principales en los ejes
    \draw[thick] (-1,-1.5pt) -- (-1,1.5pt)
      node[shift={(0.3,-0.45)}] {$-1$};
    \draw[thick] (1,-1.5pt) -- (1,1.5pt)
      node[shift={(0.25,-0.45)}] {$1$};
    \draw[thick] (-1.5pt,-1) -- (1.5pt,-1)
      node[shift={(0.05,-0.25)}] {$-i$};
    \draw[thick] (-1.5pt,1) -- (1.5pt,1)
      node[shift={(0.05,0.25)}] {$i$};

    % Marcas secundarias en los ejes
    \xaxismark{\PHI*\PHI/2}{(0.5em,0.45em)}{$\frac{\varphi^2}{2}$}
    \xaxismark{\PHI*\PHI/2}{(-0.75em,-1.4em)}{$\frac{\varphi+1}{2}$}

    \xaxismark{\PHI-1}{(-0.75em,0.45em)}{$\frac{1}{\varphi}$}
    \xaxismark{\PHI-1}{(1em,-1.3em)}{$\varphi-1$}
    
    \xaxismark{(\PHI-1)/2}{(-0.55em,0.45em)}{$\frac{1}{2\varphi}$}
    \xaxismark{(\PHI-1)/2}{(0.75em,-1.4em)}{$\frac{\varphi-1}{2}$}

    \xaxismark{(2-\PHI)/2}{(0.75em,-1.4em)}{$\frac{2-\varphi}{2}$}

    \xaxismark{-0.5}{(-0.65em,0.45em)}{$-\frac{1}{2}$}

    \xaxismark{-\PHI/2}{(0.75em,0.45em)}{$-\frac{\varphi}{2}$}

    \xaxismark{-\PHI}{(-0.75em,-1.3em)}{$-\varphi$}

    \yaxismark{sqrt(4*\PHI+3)/2}{(1em,-1em)}{$\frac{\sqrt{4\varphi+3}}{2}i$}
    \yaxismark{sqrt(4*\PHI+3)/2}{(-2.75em,0.85em)}{$\frac{\varphi}{2}\sqrt{2+\varphi}i$}

    \yaxismark{sqrt(\PHI+2)/2}{(-2.25em,-0.5em)}{$\frac{\sqrt{\varphi+2}}{2}i$}

    \yaxismark{sqrt(3-\PHI)/2}{(-1.85em,-0.85em)}{$\frac{\sqrt{3-\varphi}}{2}i$}

    \yaxismark{sqrt(3-\PHI)/(2*\PHI)}{(0.55em,1.65em)}{$\frac{\sqrt{3-\varphi}}{2\varphi}i$}
    
    \yaxismark{-sqrt(3-\PHI)/(2*\PHI)}{(-1.5em,1.25em)}{$-\frac{\sqrt{3-\varphi}}{2\varphi}i$}
    
    \yaxismark{-sqrt(3-\PHI)/2}{(0.25em,1em)}{$-\frac{\sqrt{3-\varphi}}{2}i$}

    \yaxismark{-sqrt(\PHI+2)/2}{(-2.35em,0.55em)}{$-\frac{\sqrt{\varphi+2}}{2}i$}

    \yaxismark{-sqrt(4*\PHI+3)/2}{(1.3em,1em)}{$-\frac{\sqrt{4\varphi+3}}{2}i$}
    \yaxismark{-sqrt(4*\PHI+3)/2}{(-3em,-0.85em)}{$-\frac{\varphi}{2}\sqrt{2+\varphi}i$}

    % Cotas
    \cota{P2}{P1}{$\sqrt{3-\varphi}\,$}
    \cota{P1}{P3}{$\sqrt{2+\varphi}\,$}
    \cotaround{O}{$(P3) + (P4)$}{$\varphi$}
    \cota{O}{$(P3) + (P0)$}{$\varphi-1$}
    \cotaround{P2}{$(P4) + (P0)$}{$\varphi^2$}
    \cota{P1}{P3}{$\sqrt{2+\varphi}\,$}
    \cota{P3}{$(P0) + (P1)$}{$\varphi+1$}
    \cota{O}{$(P4) + (P2)$}{$\frac{1}{\varphi}$}
    
    % Origen
    \draw (-115:0.115)  node[circle, fill=black, inner sep=1pt] {$O$};


    % Etiquetas de z^i
    \node at (4.5:1.1) {\LARGE$\zeta^0$};
    \foreach \i in {1,...,4}{
      \node at (72*\i:1.1) {\LARGE$\zeta^{\i}$};
    }

    % Etiquetas de z^i + z^(i + 1)
    \node at (36:1.75) {\Large$1+\zeta$};
    \node at (108:1.7) {\Large$\zeta+\zeta^2$};
    \node at (176.5:1.75) {\Large$\zeta^2+\zeta^3$};
    \node at (252:1.7) {\Large$\zeta^3+\zeta^4$};
    \node at (324:1.75) {\Large$\zeta^4+1$};

    % Etiquetas de z^i + z^(i + 2)
    \etiq{7.1:0.75}{$\zeta^4+\zeta$}
    \etiq{59.9:0.71}{$1+\zeta^2$}
    \etiq{153.7:0.73}{$\zeta^3+\zeta$}
    \etiq{229.5:0.48}{$\zeta^4+\zeta^2$}
    \etiq{300.1:0.71}{$\zeta^3+1$}
\end{tikzpicture}

\end{document}
